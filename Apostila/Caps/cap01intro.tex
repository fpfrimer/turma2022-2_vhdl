% Capítulo intordutório sobre VHDL
\chapter{Introdução}

Esta apostila apresenta conceitos fundamentais de VHDL utilizados na disciplina, introduzindo a base tecnológica e a linguagem de descrição. Especificamente, este capítulo aborda a definição de Dispositivos Lógicos Programáveis (PLDs), a importância das Linguagens de Descrição de Hardware (HDLs) e uma introdução ao VHDL.

\section{O que são Dispositivos Lógicos Programáveis?}

Em um futuro distante, um marceneiro chega ao seu galpão para mais um dia de trabalho. No interior do espaço, repousam alguns cubos perfeitos — cada um com exato um metro de lado — feitos de um material misterioso. O marceneiro pega um desses cubos, leva-o ao centro do galpão e, com um pequeno dispositivo em mãos, emite comandos que dão vida à matéria inerte. Diante de seus olhos, o cubo se reconfigura: transforma-se primeiro em uma mesa robusta, depois em uma cadeira ergonômica, em seguida em uma estante organizada — tudo de acordo com a necessidade daquele momento.

O segredo está na divisão de tarefas: o dispositivo armazena os \textit{blueprints} (os planos detalhados de cada móvel), enquanto o cubo é feito de um material inteligente, capaz de se moldar dinamicamente, respeitando apenas os limites de sua massa e volume.

Embora soe como ficção científica, essa metáfora ilustra o funcionamento dos \textbf{Dispositivos Lógicos Programáveis} (PLDs – \textit{Programmable Logic Devices}): componentes eletrônicos cuja arquitetura interna pode ser configurada pelo usuário para executar funções lógicas específicas, simplesmente carregando um novo “projeto” — ou seja, um novo arquivo de configuração.

No contexto dos PLDs, o ``material inteligente'' é representado por uma matriz de blocos lógicos e interconexões programáveis. Esses blocos podem ser configurados para realizar operações lógicas variadas, enquanto as interconexões permitem que os blocos se comuniquem, formando circuitos complexos. O \textit{blueprint}, por sua vez, representa o projeto digital. Esse projeto pode ser descrito tanto por diagramas esquemáticos quanto por \textbf{Linguagens de Descrição de Hardware} (HDLs).

Existem diversas categorias de PLDs. Entre as mais comuns estão os \textbf{FPGAs} (\textit{Field-Programmable Gate Arrays}) e os \textbf{CPLDs} (\textit{Complex Programmable Logic Devices}). Os FPGAs são conhecidos por sua alta densidade lógica e flexibilidade, permitindo a implementação de sistemas digitais complexos. Já os CPLDs costumam ser mais simples, possuem arquitetura menos granular e são ideais para lógicas de controle e aplicações que exigem previsibilidade temporal rigorosa.

\section{Linguagens de Descrição de Hardware}

Uma HDL (\textit{Hardware Description Language}) \textbf{não é uma linguagem de programação}. Como o próprio nome sugere, trata-se de uma linguagem utilizada para \textbf{descrever} o hardware. Diferentemente das linguagens de software (como C ou Python), que geram instruções sequenciais para um processador, as HDLs permitem que engenheiros definam a estrutura e o comportamento de circuitos eletrônicos, onde muitas coisas acontecem simultaneamente (concorrência).

Atualmente, as HDLs mais populares para configurar PLDs são o \textbf{VHDL} (\textit{VHSIC Hardware Description Language}), o \textbf{Verilog} e o \textbf{SystemVerilog}. Todas permitem a descrição de circuitos em diferentes níveis de abstração, desde o comportamental (o que o circuito faz) até o estrutural (quais portas lógicas compõem o circuito). Esta apostila foca no uso do \textbf{VHDL}.

\section{O que é VHDL?}

A sigla VHDL significa \textbf{VHSIC Hardware Description Language}, onde VHSIC é o acrônimo para \textit{Very High Speed Integrated Circuit}. Desenvolvida na década de 1980 pelo Departamento de Defesa dos Estados Unidos, a linguagem foi criada para documentar e simular o comportamento de circuitos integrados complexos. Com o tempo, o VHDL evoluiu e se tornou um padrão internacional (IEEE 1076), amplamente adotado na indústria eletrônica para o design de sistemas digitais.

Para exemplificar, considere o trecho de código \ref{cod:porta_and} escrito em VHDL, que descreve uma porta AND de duas entradas. Não se preocupe em entender todos os detalhes agora; o objetivo é apenas ilustrar a estrutura básica da linguagem.

\begin{code}
\caption{Exemplo de uma porta AND em VHDL}
\label{cod:porta_and}
\begin{lstlisting}
library ieee;
use ieee.std_logic_1164.all;

entity AND2 is
    port (
        A, B : in  std_logic;
        Y    : out std_logic
    );
end AND2;

architecture rtl of AND2 is
begin
    Y <= A and B;
end rtl;
\end{lstlisting}
\end{code}

Observe no Código \ref{cod:porta_and} a estrutura básica de um arquivo VHDL:

\begin{itemize}
    \item \textbf{Bibliotecas:} As linhas iniciais importam a biblioteca \texttt{ieee}, necessária para usar o tipo de dado padrão \texttt{std\_logic}. Outras bibliotecas podem ser incluídas conforme a complexidade do projeto.
    \item \textbf{Entidade (Entity):} Define a interface do componente, ou seja, suas entradas (\texttt{A}, \texttt{B}) e saídas (\texttt{Y}). É a "caixa preta" vista de fora. Aqui, o projetista deve prever quais sinais o componente receberá e quais ele fornecerá.
    \item \textbf{Arquitetura (Architecture):} Descreve o comportamento interno. Neste caso, a saída \texttt{Y} recebe o resultado da operação lógica AND entre \texttt{A} e \texttt{B}. É nessa região que os iniciantes em VHDL possuem maior dificuldade, pois já estão acostumados a pensar em termos de sequências de comandos comuns nas linguagens de programação. Dessa forma, o VHDL exige uma mentalidade de descrição concorrente. Na arquitetura, todas as operações são consideradas simultâneas, refletindo a natureza paralela do \textit{hardware} alvo.
\end{itemize}

Além da descrição para síntese (criação do circuito físico), o VHDL é amplamente utilizado para \textbf{simulação}. Isso permite que os desenvolvedores validem o comportamento do circuito no computador antes de implementá-lo no PLD, economizando tempo e garantindo a confiabilidade do projeto.

O código a seguir apresenta um exemplo simples de testbench em VHDL, utilizado para simular o comportamento da porta AND definida no Código \ref{cod:porta_and}.

\begin{code}
\caption{Exemplo de Testbench para a porta AND}
\label{cod:testbench_and}
\begin{lstlisting}
library ieee;
use ieee.std_logic_1164.all;    
use ieee.numeric_std.all;
entity tb_AND2 is
end tb_AND2;
architecture behavior of tb_AND2 is
    signal A, B : std_logic := '0';
    signal Y    : std_logic;
    
    component AND2
        port (
            A, B : in  std_logic;
            Y    : out std_logic
        );
    end component;
begin
    uut: AND2 port map (A => A, B => B, Y => Y);   
    process
    begin
        A <= '0'; B <= '0'; wait for 10 ns;
        A <= '0'; B <= '1'; wait for 10 ns;
        A <= '1'; B <= '0'; wait for 10 ns;
        A <= '1'; B <= '1'; wait for 10 ns;
        wait;
    end process;    
end behavior;
\end{lstlisting}    
\end{code}

No Código \ref{cod:testbench_and}, o testbench cria sinais de entrada (\texttt{A} e \texttt{B}) e conecta a unidade sob teste (UUT - \textit{Unit Under Test}), que é a porta AND definida anteriormente. O processo dentro do testbench aplica diferentes combinações de entradas, aguardando 10 nanosegundos entre cada mudança, permitindo observar a saída \texttt{Y} em resposta às entradas. O resultado da simulação pode ser visualizado em uma forma de onda, facilitando a verificação do comportamento correto do circuito, como pode ser visto na Figura \ref{fig:onda_and}.

\begin{figure}[htbp]
    \centering
    \caption{Forma de onda da simulação da porta AND}
    \label{fig:onda_and}
    \includegraphics[draft,width=0.7\textwidth]{onda_and.png}    
\end{figure}

Dessa forma, o VHDL vai possuir duas formas principais de uso:
\begin{itemize}
    \item \textbf{Síntese:} Onde o código VHDL é infere o circuito lógico que será implementado no PLD.
    \item \textbf{Simulação:} Onde o código VHDL é utilizado para validar o comportamento do circuito antes da síntese.
\end{itemize}

Essas duas vertentes são fundamentais no fluxo de projeto digital, garantindo que o design atenda aos requisitos funcionais e de desempenho antes da implementação física. No entando, cada uma dessas vertentes possui suas próprias regras e boas práticas de codificação, que serão abordadas nos capítulos seguintes. Isso acontece porque certos construtos do VHDL são adequados para simulação, mas não para síntese, e vice-versa. Por exemplo, estruturas como \texttt{wait} e \texttt{after} são úteis para simulação, mas não podem ser sintetizadas em hardware.

\section{Histórico e Evolução da Linguagem VHDL}

A linguagem VHDL (\textit{VHSIC Hardware Description Language}) possui uma origem distinta da maioria das linguagens de programação de software, tendo sido concebida inicialmente não para a execução, mas para a documentação e preservação de projetos eletrônicos complexos.

\subsection{Origens e o Programa VHSIC (Década de 1980)}
No início da década de 1980, o Departamento de Defesa dos Estados Unidos (DoD) enfrentava uma crise logística relacionada ao ciclo de vida de seus componentes eletrônicos. Com o avanço rápido da tecnologia, chips utilizados em equipamentos militares tornavam-se obsoletos rapidamente, e a falta de documentação padronizada impedia a reprodução desses componentes por novos fabricantes.

% Referência: Department of Defense. "Requirements for Hardware Description Languages". United States, 1981.
% Referência: Dewey, A. "VHSIC Hardware Description Language (VHDL) Development Program". Proceedings of the 20th Design Automation Conference, 1983.

Para solucionar este problema, o DoD iniciou o programa VHSIC (\textit{Very High Speed Integrated Circuits}). O objetivo era criar uma linguagem agnóstica de tecnologia que pudesse descrever o comportamento e a estrutura de circuitos integrados. Em 1983, um contrato foi firmado com a Intermetrics, IBM e Texas Instruments para desenvolver a primeira versão da linguagem. Uma decisão crucial de design foi basear a sintaxe na linguagem Ada, herdando desta a tipagem forte e a não-sensibilidade a letras maiúsculas/minúsculas, visando reduzir erros de ambiguidade em projetos críticos.

% Referência: IEEE Solid-State Circuits Society. "The Roots of VHDL". IEEE SSCS Magazine, 2018.

\subsection{Padronização e Evolução (IEEE)}
Embora criada para documentação, os engenheiros perceberam rapidamente que uma descrição comportamental precisa poderia ser executada por softwares de simulação. Para fomentar a adoção pela indústria civil e garantir a interoperabilidade entre ferramentas de CAD (\textit{Computer-Aided Design}), o DoD transferiu os direitos da linguagem para o IEEE (\textit{Institute of Electrical and Electronics Engineers}) em 1986.

O primeiro padrão oficial foi publicado como \textbf{IEEE Standard 1076-1987} (conhecido como VHDL-87). Desde então, a linguagem passou por revisões periódicas para incorporar novas capacidades de síntese e verificação:

\begin{itemize}
    \item \textbf{VHDL-93:} Introduziu uma sintaxe mais consistente, identificadores estendidos e melhorias na manipulação de arquivos.
    \item \textbf{IEEE 1164:} Embora não seja uma revisão da linguagem em si, a padronização do pacote \texttt{std\_logic\_1164} foi um marco fundamental. Ela introduziu o sistema lógico de 9 valores (incluindo 'Z' para alta impedância e 'X' para desconhecido), permitindo a modelagem realista de circuitos digitais.
    \item \textbf{VHDL-2000 e 2002:} Pequenas revisões que adicionaram o modificador \texttt{protected} para tipos.
    \item \textbf{VHDL-2008 (IEEE 1076-2008):} Uma grande modernização que incorporou recursos inspirados em SystemVerilog, facilitando a verificação e reduzindo a verbosidade do código para operações comuns.
    \item \textbf{VHDL-2019:} A iteração mais recente, focada em melhorias para interfaces de verificação e introspecção de tipos.
\end{itemize}

% Referência: IEEE Standard 1076-2008. "IEEE Standard VHDL Language Reference Manual". IEEE Computer Society, 2009.
% Referência: Ashenden, Peter J. "The Designer's Guide to VHDL". Morgan Kaufmann, 3rd Edition, 2008.

\subsection{Estado da Arte e Aplicação Atual}
Atualmente, o VHDL é uma das linguagens dominantes no design de hardware digital, dividindo o mercado global com o Verilog/SystemVerilog. Sua aplicação primordial migrou da documentação para a \textbf{síntese lógica} e a \textbf{simulação}.

O VHDL é amplamente empregado no desenvolvimento de FPGAs (\textit{Field-Programmable Gate Arrays}) e ASICs (\textit{Application-Specific Integrated Circuits}). Devido à sua natureza fortemente tipada e determinística, o VHDL mantém uma forte preferência em setores que exigem alta confiabilidade e segurança crítica, como as indústrias aeroespacial, automotiva, médica e de defesa, especialmente na Europa e nas Américas. Ferramentas modernas de síntese da Intel (Quartus) e AMD/Xilinx (Vivado) oferecem suporte abrangente aos padrões mais recentes, permitindo que a linguagem descreva desde portas lógicas simples até processadores \textit{soft-core} complexos e aceleradores de inteligência artificial.

% Referência: Chu, Pong P. "FPGA Prototyping by VHDL Examples". Wiley-Interscience, 2008.
% Referência: Doulos. "VHDL History and Evolution". Doulos Technical Articles, Web Resources.
