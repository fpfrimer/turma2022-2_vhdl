\documentclass[12pt]{article}
\usepackage[utf8]{inputenc}
\usepackage[brazil]{babel}
\usepackage{geometry}
\usepackage{xcolor}
\usepackage{amsmath}
\usepackage{textcomp}
\usepackage{hyperref}
\usepackage{upgreek}
\usepackage{graphicx}
\geometry{margin=2.5cm}

\title{Laboratório 3 -- Filtro de Média Móvel no Sensor HC-SR04}
\author{Prof. Felipe Walter Dafico Pfrimer \\ Disciplina de Lógica Reconfigurável}
\date{}

\begin{document}

\maketitle

\section{Introdução}

Neste laboratório, os alunos deverão implementar um \textbf{filtro de média móvel} sobre as leituras do sensor ultrassônico HC-SR04, utilizando o kit DE10-Lite. O objetivo é suavizar as variações aleatórias (ruído) nas medições de distância, produzindo uma saída mais estável e confiável. O sistema deve permitir ao usuário alternar entre a exibição do sinal \textbf{bruto} (não filtrado) e do sinal \textbf{filtrado}, usando um botão ou chave do kit.

Este laboratório complementa o Laboratório 2, exigindo que os alunos reutilizem a FSM de medição e adicionem um novo bloco de processamento: o filtro de média móvel. A Figura~\ref{fig:ma_filter} ilustra o funcionamento básico deste tipo de filtro.

\section{Objetivos}

\begin{itemize}
    \item Implementar um filtro de média móvel (Moving Average Filter) com janela de tamanho $N$ (por exemplo, $N=4$ ou $N=8$), utilizando memória circular (shift register);
    \item Integrar o filtro ao sistema de medição do sensor HC-SR04, mantendo a mesma interface de hardware e exibição;
    \item Permitir ao usuário selecionar entre o sinal bruto e o sinal filtrado, por meio de um botão ou chave do kit (ex.: KEY0 = bruto, KEY1 = filtrado);
    \item Visualizar o impacto do filtro na resposta temporal e na estabilidade da medida;
    \item Avaliar visualmente a redução de ruído e o atraso introduzido pelo filtro.
\end{itemize}

\section{Sobre o filtro de média móvel}

O filtro de média móvel é um tipo simples de \textbf{filtro digital FIR} (Finite Impulse Response) que calcula a média aritmética dos últimos $N$ valores de entrada. Ele é amplamente utilizado em sistemas embarcados por sua simplicidade e baixo custo computacional.

Matematicamente, a saída $y[n]$ é dada por:

\begin{equation}
    y[n] = \frac{1}{N} \sum_{k=0}^{N-1} x[n-k]
\end{equation}

onde:
\begin{itemize}
    \item $x[n]$ é o valor atual da amostra;
    \item $x[n-1], x[n-2], \dots, x[n-(N-1)]$ são as $N-1$ amostras anteriores;
    \item $N$ é o tamanho da janela (número de amostras consideradas).
\end{itemize}

A Figura~\ref{fig:ma_filter} mostra a estrutura típica desse filtro: uma cadeia de registradores de deslocamento (shift registers) armazena as últimas $N$ amostras, que são somadas e divididas por $N$ para gerar a saída filtrada.

\begin{figure}[htbp!]
\centering
\includegraphics[width=0.7\textwidth]{filtor.png} % Substitua pelo nome real da imagem
\caption{Estrutura de um filtro de média móvel com janela de tamanho $N$. As amostras $x[n], x[n-1], \dots, x[n-(N-1)]$ são somadas e divididas por $N$ para produzir a saída $y[n]$.}
\label{fig:ma_filter}
\end{figure}

\noindent\textbf{Observação}: Em arquiteturas sem unidade de ponto flutuante (como FPGAs), a divisão por $N$ pode ser substituída por um \textbf{deslocamento à direita} se $N$ for potência de 2 (ex.: $N=4 \rightarrow$ deslocar 2 bits; $N=8 \rightarrow$ deslocar 3 bits). Isso preserva a precisão e evita operações custosas.

\section{Instruções, considerações técnicas e dicas}

\begin{itemize}
    \item Este laboratório \textbf{depende do Laboratório 2}. Reutilize seu código anterior (FSM, contador, conversão para milímetros, exibição nos displays) como base;
    \item A atividade pode ser realizada individualmente ou em dupla;
    \item \textbf{Filtro de média móvel}:
    \begin{itemize}
        \item Implemente uma memória circular (shift register) de profundidade $N$ (recomenda-se $N=4$ ou $N=8$);
        \item A cada nova medição, insira o valor atual no shift register e remova o mais antigo;
        \item Calcule a soma das $N$ amostras e divida por $N$ (utilizando deslocamento se $N$ for potência de 2);
        \item Armazene o resultado filtrado para exibição.
    \end{itemize}

    \item \textbf{Seleção entre sinal bruto e filtrado}:
    \begin{itemize}
        \item Use um botão (ex.: KEY0) ou chave (SW0) para alternar entre os modos;
        \item Se o botão estiver pressionado (ou a chave ligada), exiba o sinal filtrado; caso contrário, exiba o sinal bruto;
        \item Idealmente, indique no display qual modo está ativo (ex.: “B” para bruto, “F” para filtrado, ou utilize um LED indicador).
    \end{itemize}

    \item \textbf{Taxa de amostragem}: mantenha a mesma taxa do Laboratório 2 (10–20 Hz) para evitar sobrecarga e garantir estabilidade do filtro;
    \item \textbf{Precisão}: como o filtro envolve soma de múltiplos valores, dimensione os registradores de soma com largura suficiente para evitar overflow (ex.: se $N=8$ e cada amostra possui 16 bits, a soma pode chegar a $8 \times 2^{16}$ — use 20 bits ou mais);
    \item \textbf{Exibição}: mostre a distância nos displays HEX0..HEX3, conforme definido no Laboratório 2. Adicione um indicador visual (LED ou caractere no display) para mostrar se o modo é bruto ou filtrado;
    \item \textbf{Análise prática}: ao testar, observe:
    \begin{itemize}
        \item Quanto ruído é eliminado?
        \item Há atraso perceptível entre o movimento do objeto e a resposta no display?
        \item O que ocorre se você aumentar ou diminuir $N$?
    \end{itemize}
    \item Entregue as questões a seguir ao professor, \textbf{por escrito}, no dia da apresentação. Não serão aceitos arquivos digitais.
\end{itemize}

\section{Questões para entrega}

\begin{enumerate}
    \item Explique como você atualiza o filtro a cada nova medição.
    \item Explique seu planejamento para resolver esse laboratório a partir do código do Laboratório 2.
\end{enumerate}

\end{document}
